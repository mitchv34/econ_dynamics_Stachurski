
% Documents
\documentclass{article}

% Packages
\usepackage[utf8]{inputenc}
\usepackage[margin = 1in, top=2cm]{geometry}% Margins
\usepackage[spanish,english]{babel}
\usepackage{apacite}
\usepackage[round]{natbib}
\usepackage{hyperref}
\usepackage{float}
\usepackage{svg}
\usepackage{setspace} % Setting the spacing between lines
\usepackage{amsthm, amsmath, amsfonts, mathtools, amssymb, bm} % Math packages 
\usepackage{graphicx}
\usepackage{pgfplots}
\usepackage{subfig} % Manipulation and reference of small or sub figures and tables
\usepackage{hyperref} % To create hyperlinks within the document
\usepackage{appendix}
\usepackage{xcolor}
\usepackage{cancel}
\usepackage{enumerate}
\usepackage[shortlabels]{enumitem}
\usepackage{optidef}
\usepackage[round]{natbib}

% Options
\setlength{\parindent}{2em}
\setlength{\parskip}{0.2em}
\spacing{1.15}

% My Enviroments
\newtheorem{defin}{Definition.}
\newtheorem{teo}{Theorem. }
\newtheorem{lema}{Lemma. }
\newtheorem{coro}{Corolary. }
\newtheorem{prop}{Proposition. }
%\theoremstyle{definition}
\newtheorem{examp}{Example. }
\newtheorem{problem}{Problem}
\newtheorem{subproblem}{}[problem]
% \numberwithin{problem}{subsection} 

% My commands
\newcommand{\card}{\operatorname{card}}
\newcommand{\qiq}{\qquad \implies \qquad}
\newcommand{\qiffq}{\qquad \iff \qquad}
\newcommand{\qaq}{\qquad \textbf{and} \qquad}
\newcommand{\qoq}{\qquad \textbf{or} \qquad}
\newcommand{\settf}{\text{ \emph{:} }}
\newcommand{\chbox}{\makebox[0pt][l]{$\square$}\raisebox{.15ex}{\hspace{.9em}}}
\newcommand{\cchbox}{\makebox[0pt][l]{$\square$}\raisebox{.15ex}{\hspace{0.1em}$\checkmark$}}

\title{Economic Dynamics Theory and Computation\newline Excercises Chapter 3}
\author{Mitchell Valdés-Bobes}
\date{\today}

\begin{document}

\maketitle

\begin{problem}
    Let $\|\cdot\|$ on $\mathbb{R}^{k}$ be a norm that generates a metric $\rho$ on $\mathbb{R}^{k}$ via $\rho(x, y):=\|x-y\|$. 
    Show that $\rho$ is indeed a metric, in the sense that it satisfies the three axioms in the definition.
\end{problem}

\begin{proof}[Answer]
    Recall the definition of norm:
    \begin{defin}
        A norm on $\mathbb{R}^{k}$ is a mapping $\mathbb{R}^{k} \ni x \mapsto\|x\| \in \mathbb{R}$ such that, for any $x, y \in \mathbb{R}^{k}$ and any $\gamma \in \mathbb{R}$
        \begin{enumerate}
            \item $\|x\|=0$ if and only if $x=0$
            \item $\|\gamma x\|=|\gamma|\|x\|$, and
            \item $\|x+y\| \leq\|x\|+\|y\|$.
        \end{enumerate}
    \end{defin}
    and the definition of metric:
    \begin{defin}
    A metric space is a nonempty set $S$ and a metric or distance $\rho: S \times S \rightarrow$ $\mathbb{R}$ such that, for any $x, y, v \in S$,
        \begin{enumerate}
            \item $\rho(x, y)=0$ if and only if $x=y$,
            \item $\rho(x, y)=\rho(y, x)$, and
            \item $\rho(x, y) \leq \rho(x, v)+\rho(v, y)$.
        \end{enumerate}
    \end{defin}

    To show that the propoerties of metric hold for $\rho$ consider:
    \begin{enumerate}
        \item $\rho(x,y)=0\iff\|x-y\|=0\iff x-y=0\iff x=y$  for any $x,y \in S$.
        \item $\rho(x,y)=\|x-y\|=\|y-x\|=\rho(y,x)$ for any $x,y \in S$
        \item $\rho(x,y)=\|x-y\|=\|(x-z)+(z-y)\|\leq\|x-z\|+ \|y-z\| = \rho(x,z) + \rho(y,z)$ for any $x,y,z\in S$
    \end{enumerate}

\end{proof}

\begin{problem}
    Let $\|\cdot\|$ be a norm on $\mathbb{R}^{k}$. Show that for any $x, y \in \mathbb{R}^{k}$ we have $\mid\|x\|-$ $\|y\| \mid \leq\|x-y\| .$
\end{problem}
\begin{proof}[Answer]
    Assume w.l.o.g. that $\|x\| \geq \|y\|$ then
    $$\big |\|x\|-\|y\|\big| = \|x\|-\|y\|$$
    and 
    $$\|x\| = \|x-y + y \| \leq \|x-y\| + \|y\| \qiq \|x\| - \|y\| \leq \|x-y\|$$
\end{proof}

\begin{problem}
    Consider the family of norms $\|\cdot\|_{p}$ defined by
$$
\|x\|_{p}:=\left(\sum_{i=1}^{k}\left|x_{i}\right|^{p}\right)^{1 / p} \quad\left(x \in \mathbb{R}^{k}\right)
$$
where $p \geq 1$, and 
$$\|x\|_{\infty}:=\max _{1 \leq i \leq k}\left|x_{i}\right|$$

Prove that $\|\cdot\|_{p}$ is a norm on $\mathbb{R}^{k}$ for $p=1$ and $p=\infty$.
\end{problem}
\begin{proof}[Answer]
    Start with $p=1$:
    \begin{enumerate}
        \item $\|x\|_1=0 \iff \sum_{i=1}^{k}\left|x_{i}\right| = 0 \iff |x_i| = 0 \: \forall i=1,\ldots,k \iff  x = 0$
        \item $\|\lambda x\|_1 = \sum_{i=1}^{k}\left|\lambda x_{i}\right| \sum_{i=1}^{k}|\lambda|\left|x_{i}\right| = |\lambda|\sum_{i=1}^{k}\left|x_{i}\right| = |\lambda| \|x\|_1$ for all $x\in \mathbb{R}^k$, $lambda\in \mathbb{R}$
        \item $\|x + y\|_1 = \sum_{i=1}^{k}\left|x_{i} + y_i\right|$ since $|x_i + y_i|\leq |x_i| + |y_i|$ for all $i=1,\ldots, k$ then $\|x + y\|_1\leq \|x \|_1 + \| y\|_1$ for all $x,y\in \mathbb{R}^k$
    \end{enumerate}
    Then the proof is complete.

    For the case when $p=\infty$:
    \begin{enumerate}
        \item $\|x\|_\infty=0 \iff \max _{1 \leq i \leq k}\left|x_{i}\right| = 0 \iff 0 \leq \left|x_{i}\right| \leq  \max _{1 \leq i \leq k}\left|x_{i}\right|  = 0 \:  \forall i=1,\ldots,k \iff  x = 0$ 
        \item $\|\lambda x\|_\infty = \max _{1 \leq i \leq k}\left|\lambda x_{i}\right| =\max _{1 \leq i \leq k}\left|\lambda| |x_{i}\right| = |\lambda| \max _{1 \leq i \leq k}\left|x_{i}\right| = |\lambda| \|x\|_\infty$ for all $x\in \mathbb{R}^k$, $lambda\in \mathbb{R}$
        \item $\| x + y\|_\infty = \max _{1 \leq i \leq k}\left|x_{i} + y_{i}\right|=\left|x_{j} + y_{j}\right|$ for some $j\in 1,\ldots , k$; then
              $\left|x_{j} + y_{j}\right| \leq |x_j| + |y_j| \leq \max _{1 \leq i \leq k}\left|x_{i}\right| + \max _{1 \leq i \leq k}\left|y_{i}\right| = \|x\|_\infty+\|y\|_\infty$  for all $x,y\in \mathbb{R}^k$
    \end{enumerate}
\end{proof}

\begin{problem}
    Let $\left(x_{n}\right)$ and $\left(y_{n}\right)$ be sequences in $S$. Show that if $x_{n} \rightarrow x \in S$ and $\rho\left(x_{n}, y_{n}\right) \rightarrow 0$, then $y_{n} \rightarrow x$
\end{problem}

\begin{proof}
    Recall that:
    $$x_n \to x \qiffq \forall\varepsilon_1 > 0 \: \exists N_1\in \mathbb{N} \mid \rho(x_n,x)< \varepsilon_1$$

    and 

    $$\rho(x_n,y_n) \to 0 \qiffq \forall\varepsilon_1 > 0 \: \exists N_2\in \mathbb{N} \mid |\rho(x_n,y_n)|< \varepsilon_2$$

    Then consider $\rho(y_n,x)$ and note that
    $$\rho(y_n,x)<\rho(x_n,y_n) + \rho(x_n,x) \leq \varepsilon = \varepsilon_1 + \varepsilon_2 \qquad \forall \:n > \max\{N_1,N_2\}$$
\end{proof}

\begin{problem}
    Let $\left(x_{n}\right) \subset S$ and $x \in S$. Show that $x_{n} \rightarrow x$ if and only if for all $\varepsilon>0$, the ball $B(\varepsilon ; x)$ contains all but finitely many terms of $\left(x_{n}\right)$.
\end{problem}

\begin{proof}[Answer]
    It follows from the definition of convergence that:
    $$x_n \to x \qiffq \forall \varepsilon > 0 \: \exists N \in \mathbb{N} \mid \rho(x_n,x)< \varepsilon \qiffq x_n \in B(\varepsilon; x) \forall n \geq N$$
    This means that only the $N$ first terms of the sequence are \textbf{not} in the ball $B(\varepsilon;x)$.
\end{proof}

\begin{problem}
    Show that every convergent sequence in $S$ is also bounded.
\end{problem}

\begin{proof}[Answer]
    Recall that a subset $E$ of $S$ is called bounded if $E \subset B(n ; x)$ for some $x \in S$ and some (suitably large) $n \in \mathbb{N}$. A sequence $\left(x_{n}\right)$ in $S$ is called bounded if its range $\left\{x_{n}: n \in \mathbb{N}\right\}$ is a bounded set.

    Assume that $(x_n)$ is a convergent sequence then and select any $\varepsilon>0$ we know that thre is $N$ such that $x_n\in B(\varepsilon; x)$ for all $n>N$.

    For all $x_n$ with $n\leq N$ define $\delta_n = \rho(x_n, x) + 0.1$ then and $\bar{\varepsilon} = \max\{\delta_1,\ldots,\delta_N,\varepsilon\}$. It is clear by construction that $\rho(x_n,x)\leq \bar{\varepsilon}$ thus $\{x_n\}_{n=1}^\infty \subset B(\bar{\varepsilon; x})$, therefore the sequence is bounded.
\end{proof}

\begin{problem}
    Show that for $\left(x_{n}\right) \subset S, x_{n} \rightarrow x$ for some $x \in S$ if and only if ever subsequence of $\left(x_{n}\right)$ converges to $x$.
\end{problem}

\begin{proof}
    Sufficiency is trivial since the whole sequence is a subsequence of itself.

    To show necessity consider $x_n\to x$ and a subsequence $\{x_{k_n}\}\subset\{x_{n}\}$ since subsequences are defined by increasing functions we know that if the term $x_n$ is "behind" $x_m$ in the original seuqnces then and those terms are in a subsequence then the ordering is preserved. Then for any $varepsilon>0$ we have that there is a term $x_N$ such that for any term  $x_n$ with $n>N$ $\rho(x_n, x_n)<\varepsilon$. Then let consider $x_{k_{N}}$ the first term in the subsequence that corresponds to $x_N$ in the original sequence this means that any terms that follow $x_{k_{N}}$ is close enough to $x$ so the definition of convergence holds.

\end{proof}

\begin{problem}
    Let $f ( x, y ) = x^2 + y^2$ . Show that $f$ is a continuous function from
    $(\mathbb{R}^2 , d_2 )$ into $(\mathbb{R} , | \cdot |)$.
    
\end{problem}

\begin{proof}[Answer]
    \begin{align*}
        (x_n,y_n)\to (x,y)  & \qiffq x_n \to x \qaq y_n \to y \\ &\qiffq x_n^2 \to x^2 \qaq y_n^2 \to y^2\\ 
                            &\qiffq \forall \frac{\varepsilon}{2} \exists N \in \mathbb{N}\:\mid \forall n>N \quad |x_n^2 - x^2|<\frac{\varepsilon}{2} \qaq |y_n^2 - y^2|<\frac{\varepsilon}{2}\\
                            &\qiffq \forall \varepsilon>0 \quad |(x_n^2+y_n^2) - (x^2+y^2)|\leq |x_n^2 - x^2| + |y_n^2 - y^2| < \varepsilon\\
                            &\qiffq f(x_n,y_n)\to f(x,y)
    \end{align*}
\end{proof}

\begin{problem}
    Let $f$ and $g$ be as above, and let $S$ be a metric space. Show that if $f$ and $g$ are continuous, then so are $f+g$ and $f g$.
\end{problem}

\begin{proof}[Answer]
    
\end{proof}

\begin{problem}
    A function $f: S \rightarrow \mathbb{R}$ is called upper-semicontinuous (usc) at $x \in S$ if, for every $x_{n} \rightarrow x$, we have $\lim \sup _{n} f\left(x_{n}\right) \leq f(x) ;$ and lower-semicontinuous (lsc) if, for every $x_{n} \rightarrow x$, we have $\lim \inf _{n} f\left(x_{n}\right) \geq f(x)$. Show that $f$ is usc at $x$ if and only if $-f$ is $\operatorname{lsc}$ at $x$. Show that $f$ is continuous at $x$ if and only if it is both usc and lsc at $x$. 
\end{problem}

\begin{proof}[Answer]
    Recall that
    $$\liminf a_{n}:=\lim _{n \rightarrow \infty} \inf _{k \geq n} a_{k} \qaq \lim \sup a_{n}:=\lim _{n \rightarrow \infty} \sup _{k \geq n} a_{k}$$

    Let $f$ be an uper-semicontinuous function then:
    $$x_n \to x \qiq   \lim \sup _{n} f\left(x_{n}\right) \leq f(x)$$
    Note that  $$\forall n \qquad  \sup_{k\geq n} f(x_k) = -\inf_{k\geq n} -f(x_k) $$
    Therefore
    $$\lim \sup _{n} f\left(x_{n}\right) \leq f(x) \qiq -\lim \sup _{n} f\left(x_{n}\right)= \lim \inf _{n} f\left(x_{n}\right)\geq -f(x)$$
    
\end{proof}

\end{document}