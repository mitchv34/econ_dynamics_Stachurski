% Documents
\documentclass{article}

% Packages
\usepackage[utf8]{inputenc}
\usepackage[margin = 1in, top=2cm]{geometry}% Margins
\usepackage[spanish,english]{babel}
\usepackage{apacite}
\usepackage[round]{natbib}
\usepackage{hyperref}
\usepackage{float}
\usepackage{svg}
\usepackage{setspace} % Setting the spacing between lines
\usepackage{amsthm, amsmath, amsfonts, mathtools, amssymb, bm} % Math packages 
\usepackage{graphicx}
\usepackage{pgfplots}
\usepackage{subfig} % Manipulation and reference of small or sub figures and tables
\usepackage{hyperref} % To create hyperlinks within the document
\usepackage{appendix}
\usepackage{xcolor}
\usepackage{cancel}
\usepackage{enumerate}
\usepackage[shortlabels]{enumitem}
\usepackage{optidef}
\usepackage[round]{natbib}

% Options
\setlength{\parindent}{2em}
\setlength{\parskip}{0.2em}
\spacing{1.15}
\setcounter{section}{3}

% My Enviroments
\newtheorem{defin}{Definition.}
\newtheorem{teo}{Theorem. }
\newtheorem{lema}{Lemma. }
\newtheorem{coro}{Corolary. }
\newtheorem{prop}{Proposition. }
%\theoremstyle{definition}
\newtheorem{examp}{Example. }
\newtheorem{problem}{Problem}[section]
\newtheorem{subproblem}{}[problem]
% \numberwithin{problem}{subsection} 

% My commands
\newcommand{\card}{\operatorname{card}}
\newcommand{\qiq}{\qquad \implies \qquad}
\newcommand{\qiffq}{\qquad \iff \qquad}
\newcommand{\qaq}{\qquad \textbf{and} \qquad}
\newcommand{\qoq}{\qquad \textbf{or} \qquad}
\newcommand{\settf}{\text{ \emph{:} }}
\newcommand{\chbox}{\makebox[0pt][l]{$\square$}\raisebox{.15ex}{\hspace{.9em}}}
\newcommand{\cchbox}{\makebox[0pt][l]{$\square$}\raisebox{.15ex}{\hspace{0.1em}$\checkmark$}}

\title{Economic Dynamics Theory and Computation\newline Excercises Chapter 4}
\author{Mitchell Valdés-Bobes}
\date{\today}

\begin{document}

\maketitle
\section{Deterministic Dynamical Systems}

\subsection{The Basic Model}

\begin{problem}\label{problem_1}
    Show that if $(S, h)$ is a dynamical system, if $x^{\prime} \in S$ is the limit of some trajectory (i.e., $h^{t}(x) \rightarrow x^{\prime}$ as $t \rightarrow \infty$ for some $\left.x \in S\right)$, and if $h$ is continuous at $x^{\prime}$, then $x^{\prime}$ is a fixed point of $h$.    
\end{problem}

\begin{proof}[Answer]
    Consider the sequence $\{x_t\}\subset S$ defined as $x_t = h^t(x)$ since $h$ is a continuous function we have:
    $$x_t = h^t(x) \to x' \qiq h(x_t) = h^{t+1}(x) \to h(x')$$
    Since the limit of a sequence must be unique we have that
    $$\boxed{h(x')=x'}$$
\end{proof}

\begin{problem}
    Prove that if $h$ is continuous on $S$ and $h(A) \subset A$ (i.e., $h$ maps $A \rightarrow A$ ), then $h(\operatorname{cl} A) \subset \operatorname{cl} A$
\end{problem}

\begin{proof}[Answer]
    Let $x\in h(\operatorname{cl}A)$ this means that there is $x'\in \operatorname{cl}A$ such that $h(x')=x$. If $x'\in A$ then $h(x')=x\in A$ if $x'\not\in A$ then there is $\{x_t\}\subset A$ such that $x_t\to x'$. Note that since $h$ is continuous then $h(x_t)\to h(x')=x$ and since $h(x_t)\in A$ for all $t$ then $x\in \operatorname{cl}A$.
\end{proof}

\begin{problem}
    Prove that $x^{*}$ is locally stable if and only if there exists an $\epsilon>0$ such that $B\left(\epsilon, x^{*}\right) \subset \Lambda\left(x^{*}\right)$
\end{problem}

\begin{proof}[Answer]
    This problem is straightforward we get sufficiency since $B(\varepsilon; x^*)$ is an oppen set and necessity by definition of an open set that mus include a ball of radious $\varepsilon>0$ for some $\varepsilon$.
\end{proof}

\begin{problem}
    Prove that if $x^{*}$ is a fixed point of $(S, h)$ to which every trajectory converges, then $x^{*}$ is the only fixed point of $(S, h)$.
\end{problem}

\begin{proof}[Answer]
    Prove it by contradiction suppose that there are $x^{*} \neq x^{**}$ fixed points, and that \textbf{every} trayectory convertges to $x^*$, but this must be a contradiction since $x^{**}$ is a fixed point implies that the trayectory $h^t(x^{**}) = x^{**}\to x^{**}$. 
\end{proof}

\begin{problem}
    Prove \textbf{Lemma 4.1.7}:
    
    If $h$ is a map with continuous derivative $h^{\prime}$ and $x^{*}$ is a fixed point of $h$ with $\left|h^{\prime}\left(x^{*}\right)\right|<1$, then $x^{*}$ is locally stable.

\end{problem}

\begin{proof}[Answer]
    Consider $x^*$ is a fixed point of $h$; by the definition of derivative we have that for any $x_n \to x^*$: 

    $$\left |h'(x^*)\right| = \lim_{n\to \infty} \frac{\rho(h(x_n) , h(x^*))}{\rho(x_n - x^*)} < 1 $$

    By the definition of limit we have that for every $\varepsilon>0$ there are $N_1$ and $N_2$ such that $\rho(x_n , x^*)<\varepsilon$ for every $n>N_1$ and  $$\lim_{t\to \infty} \frac{\rho(h(x_n) , h(x^*))}{\rho(x_t, x^*)} < 1 $$ for every $n>N_2$. Define $N = \max\{N_1,N_2\}$ and we have that:
    $$\lim_{n\to \infty} \frac{\rho(h(x_n) , h(x^*))}{\rho(x_n , x^*)} < 1 \qiq \rho(h(x_n) , h(x^*)) < \rho(x_n , x^*) < \varepsilon  $$

    therefore $h(x_n) \to h(x^*)=x^*$. We have proved that for any $x$ \textit{"close enough"} to $x^*$ $h(x)\to x^*$. Pick any such point ($x\in B(\varepsilon; x^*)$) 

    Also note that for any $t>0$
    $$\rho(h^t(x),x^*)=\rho(h(h^{t-1}(x)), x^*)<\rho(h^{t-1}(x), x^*)$$

    We can define a sequence $\varepsilon_t$ such that $\varepsilon_1 = \varepsilon$, $\varepsilon_t \to 0$ and $$0\leq \rho(h^{t}(x), x^*) < \varepsilon_t \qiq  \rho(h^{t}(x), x^*)\to 0 \qiq h^t(x) \to x^* $$
    
    Therefore $x^*$ is locally stable.

\end{proof}

\begin{problem}
    A dynamical system $(S, h)$ is called Lagrange stable if every trajectory is precompact in $S$. In other words, the set $\left\{h^{n}(x): n \in \mathbb{N}\right\}$ is precompact for every $x \in S $(i.e every subsequence of the trajectory has a convergent subsubsequence). Show that if $S$ is a closed and bounded subset of $\mathbb{R}^{n}$, then $(S, h)$ is Lagrange stable for any choice of $h$.
\end{problem}

\begin{proof}[Answer]
    Since $S$ is a bounded and closed subset of $\mathbb{R}^n$ then for any $x$ $\left\{h^{n}(x): n \in \mathbb{N}\right\}\subset S$ implies that the sequence $x_n = h^n(x)$ is bounded, then by \textbf{Bolzano-Weierstrass theorem} it must have a convergent subsequence and since $S$ is closed that subsequence must converge to an element of $S$.
\end{proof}

\begin{problem}
    Give an example of a dynamical system $(S, h)$ where $S$ is unbounded but $(S, h)$ is Lagrange stable.
\end{problem}

\begin{proof}[Answer] consider the followign two examples.
    \begin{itemize}
        \item Trivial example $S = \mathbb{R}$ and $h(x) = c\in \mathbb{R}$ for all $x\in \mathbb{R}$.
        \item Less trivial, same $S$ but $h(x) = x/c$ for some $|c|>1$. Note that $h^t(x) = x/c^t \to 0$ for any $x\in \mathbb{R}$. 
    \end{itemize}

\end{proof}

\begin{problem}
    Let $S=\mathbb{R}$, and let $h: \mathbb{R} \rightarrow \mathbb{R}$ be an increasing function, in the sense that if $x \leq y$, then $h(x) \leq h(y)$. Show that every trajectory of $h$ is a monotone sequence in $\mathbb{R}$ (either increasing or decreasing).
\end{problem}

\begin{proof}[Answer]
    Consider an $x\in \mathbb{R}$ such that $h(x)\leq x$ then $h^2(x) \leq h(x)$ iterating forward $h^(t+1)(x) \leq h^t(x)$ for all $t>0$ therefore the trayectory is decreasing. Alternatively if $x\leq h(x)$ we get an increasing trayectory.
\end{proof}

\begin{problem}
    Now order points in $\mathbb{R}^{n}$ by setting $x \leq y$ whenever $x_{i} \leq y_{i}$ for $i$ in $\{1, \ldots, n\}$ (i.e., each component of $x$ is dominated by the corresponding component of $y$ ). Let $S=\mathbb{R}^{n}$, and let $h: S \rightarrow S$ be monotone increasing. (The definition is the same.) Show that the same result no longer holds $-h$ does not necessarily generate monotone trajectories.
\end{problem}

\begin{proof}[Answer]
    \textcolor{red}{MISSING!!!}
\end{proof}

\begin{problem}\label{ex10}
    Let $S=(\mathbb{R},|\cdot|)$ and $h(x)=a x+b$. Prove that
$$
h^{t}(x)=a^{t} x+b \sum_{i=0}^{t-1} a^{i} \quad(x \in S, t \in \mathbb{N})
$$
(Hint: Use induction.) From this expression, prove that $(S, h)$ is globally stable whenever $|a|<1$, and exhibit the fixed point.
\end{problem}

\begin{proof}[Answer]
    Since $$h(x) = h^1(x) = ax + b = a^1 x + b\sum_{i=0}^0a^i $$ we have our base case.
    Now, assume that for $t = k$
    $$
    h^{k}(x)=a^{k} x+b \sum_{i=0}^{k-1} a^{i}
    $$
    We will apply $h$ to $h^k$ to obtain $h^{k+1}$ and check that the formula still holds:
    
    \begin{align*}
        h^{k^+1}(x)=h(h^{k}(x))&= a\left(a^{k} x+b \sum_{i=0}^{k-1} a^{i}\right) + b\\
                                &= a^{k+1}x + b\sum_{i=0}^{k-1} a^{i+1} + b a^0\\
                                &=a^{k+1}x + b\sum_{i=0}^{k} a^{i}\\
    \end{align*}

    Then if $|a|<1$ $$\lim_{t\to \infty} h^t(x) = b \sum_{i=0}^\infty a^i = \frac{b}{1-a}\qquad(\forall x \in S)$$
\end{proof}


\begin{problem}
    Show that the condition $|a|<1$ is also necessary, in the sense that if $|a| \geq 1$, then $(S, h)$ is not globally stable. Show, in particular, that $h^{t}\left(x_{0}\right)$ converges to $x^{*}:=b /(1-a)$ only if $x_{0}=x^{*}$
\end{problem}

\begin{proof}[Answer]
    \textcolor{red}{MISSING}
\end{proof}


\begin{problem}
    Let $(S, h )$ be as in \ref{ex10} Using Banach's Fixed Point Theorem, prove that $(S, h )$ is globally stable whenever $| a | < 1$.
    
\end{problem}

\begin{proof}[Answer]
    Recall:
    \begin{teo}[Banach]
        Let $T: S \rightarrow S$, where $(S, \rho)$ is a complete metric space. If $T$ is a uniform contraction on $S$ with modulus $\lambda$, then $T$ has a unique fixed point $x^{*} \in S .$ Moreover for every $x \in S$ and $n \in \mathbb{N}$ we have $\rho\left(T^{n} x, x^{*}\right) \leq \lambda^{n} \rho\left(x, x^{*}\right)$, and hence $T^{n} x \rightarrow x^{*}$ as $n \rightarrow \infty$
    \end{teo}
\end{proof}

Consider $$\rho(h(x),h(y)) = \rho(ax + b, ay + b) = \left(a\left(y+\frac{b}{a}\right),a\left(y+\frac{b}{a}\right)\right) = |a||x-y| = |a|\rho(x,y)$$
Then if $|a|<1$ select $|0<\lambda < |a|<1$ and we have $$\rho(h(x),h(y))<\lambda \rho(x,y)$$

We have showed that $T:S\to S$ where $Tx = h(x)$ is a uniform contraction on $S$ therefore it has a unique fixed point $x^*$ and $h^n(x)\to x^*$ as $n\to \infty$.

\begin{problem}
    Let $S:=(0, \infty)$ with $\rho(x, y):=|\ln x-\ln y| .$ Prove that $\rho$ is a metric on $S$ and that $(S, \rho)$ is a complete metric space. Consider the growth model $k_{t+1}=$ $h\left(k_{t}\right)=s A k_{t}^{\alpha}$, where $s \in(0,1], A>0$ and $\alpha \in(0,1) .$ Convert this into a dynamical system on $(S, \rho)$, and prove global stability using theorem Banach's Fixed Point Theorem.
\end{problem}

\begin{proof}[Answer]
    We start by proving the $\rho$ is a metric:
    \begin{enumerate}[(i)]
        \item $\rho(x,y) = 0 \iff |\ln x - \ln y|=0 \iff \ln x = \ln y \iff \exp(\ln x) = \exp(\ln y)\iff x = y$
        \item $\rho(x,y) =  |\ln x - \ln y| =  |\ln y - \ln x| = \rho(y,x) \qquad \forall x,y\in S$
        \item $\rho(x,y) =  |\ln x - \ln y| = |(\ln x - \ln z) - (\ln y -\ln z)| \leq |\ln x - \ln z| + |\ln y -\ln z| = \rho(x,z) + \rho(y,z)\qquad \forall x,y,z\in S$
    \end{enumerate}

    Next we show that $(S,\rho)$ is a Complete Space. Consider a function $f:(S,\rho)\to (\mathbb{R}, |\cdot|)$ such that $f(x) = \ln x$  and a Cauchy sequence $\{x_n\}$:
    
    $$\forall \varepsilon> 0 \quad \exists N \in \mathbb{N} \mid \rho(x_n,x_m) = |\ln x - \ln y| = |f(x) - f(y)| < \varepsilon \quad \forall n,m > N $$

    then $\{f(x_n)\}$ is a Cauchy sequence in $(\mathbb{R},|\cdot|)$ which is a compelte space thus $f(x_n) \to y = \ln x $ for some $x\in (0, \infty)$. Using the definition of limit we have:
    
    $$\forall \varepsilon > 0 \quad \exists N\in \mathbb{N}\mid |f(x_n) - y| = |\ln x_n - \ln x| = \rho(x_n,x) < \varepsilon     \quad \forall n> N$$

    We have showed that $x_n \to x$ in $(S, \rho$), therefore $(S, \rho$) is complete.

    Let $h: S \to S$ where $h(x) = sAx^\alpha$ consider $x,y\in S$:
    $$\rho(h(x),h(y)) = |\ln (sAx^\alpha) - \ln (sAx^\alpha)| = |\cancel{\ln s} + \cancel{\ln A} + \alpha \ln x - \cancel{\ln s}  - \cancel{\ln A} - \alpha \ln y| = |\alpha(\ln x - \ln y)| = \alpha \rho(x,y)$$

    Since $\alpha \in (0,1)$ we can select $0<\lambda < \alpha < 1$ to get $$\rho(h(x), h(y))< \lambda \rho(x,y)$$ Since we have proved that $h$ is a uniform contraction with modulus $\lambda$ then we can apply Banach's Fixed Point Theorem to prove that $h$has only one fixed point $x^*$ and that $\lim_{t\to \infty} h^t(x) = x^*$ thus concluding the prove that the dynamical system $(S, h)$ is globally stable.


\end{proof}


\begin{problem}\label{prob414}
    Consider the mapping $h(x)=E x+b$ where $E$ is an $n \times n$ matrix and $b \in \mathbb{R}^{n}$, let $\|\cdot\|$ be any norm on $\mathbb{R}^{n}$, and define
$$
\lambda:=\max \left\{\|E x\|: x \in \mathbb{R}^{n},\|x\|=1\right\}
$$
If you can, prove that the maximum exists. Using the properties of norms and linearity of $E$, show that $\|E x\| \leq \lambda\|x\|$ for all $x \in \mathbb{R}^{n}$. Show in addition that if $\lambda<1$, then $\left(\mathbb{R}^{n}, h\right)$ is globally stable.
\end{problem}

\begin{proof}[Answer]
    We start by showing that the maximum exists. First define the set $A = \{x\in \mathbb{R}^n\mid \| x\| = 1 \}$, clearly $A$ is a bounded and closed, therefore compact (by Heine-Borel); next consider the continuous function $f:A\to \mathbb{R}$ defined as $f(x) = \|E x\|$, since $f$ is a continous function on a compact set it attains it's supremum.

    Next we will show that $h$ is a uniform contraction of modulus $\lambda$. $$\|Ex\| = \frac{\|x\|}{\|x\|}\|Ex\| =\left\| E \left(\frac{x}{\|x\|} \right)\right\|  \|x\| \leq \lambda \|x\| $$

    Showing that $(\mathbb{R}, h)$ is globally stable is just an aplicaction of Banach's Fixed Point Theorem.
\end{proof}


\begin{problem}\label{prob415}
    Prove that Long and Plosser's system is stable in the following way:
Let $A=\left(a_{i j}\right)$ be an $n \times n$ matrix where the sum of any of the rows of $A$ is strictly less than 1 (i.e., $\max _{i} \alpha_{i}<1$, where $\left.\alpha_{i}:=\sum_{j}\left|a_{i j}\right|\right)$. Using the norm $\|\cdot\|_{\infty}$ in \ref{prob414}, show that for $A$ we have $\lambda<1$. Now argue that in Long and Plosser's model, $\left(y_{t}\right)$ converges to a limit $y^{*}$, which is independent of initial output $y_{0}$, and, moreover, is the unique solution to the equation $y^{*}=A y^{*}+b .$
\end{problem}

\begin{proof}[Answer]
    Consider $x\in S$ such that $\|x\|_\infty = \max_{i = 1 \ldots n } |x_i| = 1$ and $\|Ax\|_\infty$ since $x = (x_1, \ldots, x_n) = \sum_{i=1}^n x_i e_i$ where $e_i$ is the canonical base of $\mathbb{R}^n$ then

    $$\|Ax\|_\infty  =  \left\|A\sum_{i=1}^n x_i e_i\right\| \leq \sum_{i=1}^n|x_i|\|Ae_i\|_\infty \leq \|x\|_\infty \sum_{i=1}^{n} \|Ae_i\|_\infty$$

    Notice that $$Ae_i = \left(\begin{array}{c} a_{i1} \\ \ldots \\ a_{in}\end{array}\right) =  \sum_{j=1}^n a_{ij}e_j $$

    then $$\|Ae_i\|_\infty \leq \sum_{j=1}^n |a_{ij}|\cancelto{1}{\|e_j\| } \leq  \max _{i} \alpha_{i} < 1 $$

    Thus we have that $$\lambda:=\max \left\{\|A x\|: x \in \mathbb{R}^{n},\|x\|=1\right\} < 1$$

    Now consider $$h(y) = Ay+ b $$ we know that 
    \begin{align*}
    h^{t}(y_0) = A^t y_0 + b\sum_{i=0}^{t-1}A^i
    \end{align*}

    First note that $$0\leq \lim_{t\to\infty} \|A^t y_0\| \leq \lim_{t\to\infty} \|y_0\| \left\|A^t\left(\frac{y}{\|y\|}\right)\right\| \leq \lim_{t\to \infty} \lambda^t\|y_0\| = 0$$
    therefore we have showed that the limit does not depend on the initial output $y_0$.

    Next consider $y^* = Ay^* + b$ and 
    \begin{align*}
    y^* -  b\sum_{i=0}^{t-1}A^i &= Ay^* + b -  b\sum_{i=0}^{t-1}A^i = Ay^* - b\sum_{i=1}^{t-1}A^i = A\left(y^* - b\sum_{i=1}^{t-1}A^{i-1}  \right) = A\left(y^* - b\sum_{i=0}^{t-2}A^{i}  \right) \\
        &=A\left(Ay^* + b -  b\sum_{i=0}^{t-1}A^i\right)=A\left(Ay^* - b\sum_{i=1}^{t-2}A^i\right) = A^2\left(y^* - b\sum_{i=1}^{t-2}A^{i-1}  \right) = A\left(y^* - b\sum_{i=0}^{t-3}A^{i}  \right) \\
        &\ldots\\
        &=A^{t-1}(y^* - b) = A(Ay^* +b - b) = A^t y^*
    \end{align*}

    Finally $$\lim_{t\to\infty}\|h^t(y_0) - y^*\|_{\infty} = \|\lim_{t\to\infty} h^t(y_0) - y^*\|_{\infty} = \left\|\lim_{t\to\infty}y^* -  b\sum_{i=0}^{t-1}A^i\right\|_{\infty} = \|\lim_{t\to\infty} A^t y^*\|_{\infty}= \lim_{t\to\infty} \|A^t y^*\|_{\infty} = 0$$

    Therefore $$h^t(y_0) \to y^* \qquad \forall y_0 \in \mathbb{R}^n $$
\end{proof}

\begin{problem}
    Let $B=\left(b_{i j}\right)$ be an $n \times n$ matrix where the sum of any of the columns of $B$ is strictly less than 1 (i.e., $\max _{j} \beta_{j}<1$, where $\left.\beta_{j}:=\sum_{i}\left|b_{i j}\right|\right) .$ Using the norm $\|\cdot\|_{1}$ in \ref{prob414}, show that for $B$ we have $\lambda<1$. Conclude that if $h(x)=B x+b$, then $\left(\mathbb{R}^{n}, h\right)$ is globally stable.
\end{problem}

\begin{proof}[Answer]
    Follows for Problem \ref{prob415}
\end{proof}

\begin{problem}
    Suppose that $h$ is uniformly contracting on complete space $S$, so $(S, h)$ is globally stable. Prove that if $A \subset S$ is nonempty, closed and invariant under $h$ (i.e., $h(A) \subset A)$, then the fixed point of $h$ lies in $A$.
\end{problem}

\begin{proof}[Answer]
    From Banach's Fixed Point Theorem $h$ has a unique fixed point $x^*$ and $h^n(x)\to x^*$ for any $x \in S$ in particular select $x\in A$ and define $x_n = h^n(x)$, since $A$ is invariant under $h$ then $\{x_n\}\subset A$ and since $x_n \to x^*$ and $A$ is closed then $x^*\in A$. 
\end{proof}

\end{document}